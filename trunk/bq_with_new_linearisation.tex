\documentclass[twoside]{article}
\usepackage{aistats2e}

% For figures
%\usepackage{\ellraphicx} % more modern
%\usepackage{epsfig} % less modern
\usepackage{preamble}
\usepackage{natbib}




\usepackage{usbib}
\bibliographystyle{myusmeg-a}
\renewcommand{\citenamefont}[1]{\textsc{\MakeLowercase{#1}}}
\renewcommand{\bibnamefont}[1]{\textsc{#1}}
\renewcommand*{\bibname}{biblography}

%\DeclareCaptionType{copyrightbox}

\begin{document}

\twocolumn[

\aistatstitle{New linearisation for Bayesian Quadrature}

\aistatsauthor{ 
%Michael A. Osborne \And Roman   Garnett
}
\aistatsaddress{ 
% Department of Engineering Science \\
% University of Oxford \\
% Oxford OF1 3PJ, UK\\
% \url{mosb@robots.ox.ac.uk} 
\And 
}
]

\begin{abstract}%   <- trailing '%' for backward compatibility of .sty file
\end{abstract}


\section{Introduction}



\section{Bayesian Quadrature} \label{sec:bq}

% Note that maximum likelihood is also subject to issues. $\p{D}{x,I}$, how come (known) $I$ is on the right and (known) $D$ is on the left? 

\emph{Bayesian quadrature} (\bq) \citep{BZHermiteQuadrature,BZMonteCarlo} is a means of performing Bayesian inference about the value of a potentially nonanalytic integral, \begin{equation}\label{eq:bq_integral}
\inty{\ell}\deq \int \ell(\vx)~ q(\vx)~\ud \vx\,.
\end{equation}
Here $\ell$ is typically an expensive, multimodal or unknown function and $q$ is a simpler function, often an input density. 
Quadrature involves evaluating $\ell(\vx)$ at a matrix of sample points $F$, giving $\v{\ell}\deq \ell(X)$. Often this evaluation is computationally expensive; the consequent sparsity of samples introduces uncertainty about the function $\ell$ between them, and hence uncertainty about the integral $\inty{\ell}$.

Most work on \bq chooses a \gp prior for $\ell$, with zero mean and the Gaussian covariance 
\begin{align} \label{eq:Gaussian_cov_fn}
\text{cov}\bigl(\ell(f),\ell(f')\bigr)=k(\vx,\vx') & \deq \rho^2 \N{\vx}{\vx'}{\Omega}\,.
\end{align} 
The choice of covariance is coupled with the choice of $q$: we'll later see that we require that $\int k(\vx',\vx) q(\vx) \ud\vx$ is tractable.
The Gaussian covariance \eqref{eq:Gaussian_cov_fn}, has hyperparameters $\rho$, along with those parameterising the input covariance $\Omega$. In this, as in previous work, we'll take $\Omega$ as diagonal.  We'll fit these hyperparameters are using type two maximum likelihood. 

Variables possessing a multivariate Gaussian distribution are jointly Gaussian distributed with any affine transformations of those variables. Because integration is affine, we can hence use computed samples $\v{\ell}$ to perform analytic Gaussian process inference about the value of integrals over $\ell(x)$, such as $\inty{\ell}$. The mean estimate for $\inty{\ell}$ given $\v{\ell}$ is
%
\begin{align} 
&
\mean{\inty{\ell}}{\v{\ell}} 
\nonumber\\
& =\iint \inty{\ell}~\p{\inty{\ell}}{\ell}~\p{\ell}{\v{\ell}}~\ud \inty{\ell}~\ud g                                                                                                                                                               \nonumber\\
&
 =\iint \inty{\ell}\,\delta\bigl(\inty{\ell}- {\textstyle\int \ell(\vx)\,p(\vx)\,\ud \vx}\bigr)
\nonumber\\
& \hspace{4cm}
\N{\ell}{m_{\ell|X}}{C_{\ell|X}}\,\ud \inty{\ell}\,\ud g 
\nonumber\\
&
 = \int m_{\ell|X}(\vx)~q(\vx)~\ud \vx
\,, \label{eq:mean_inty}
\end{align}
and, similarly, the variance is
\begin{align} 
V(\inty{\ell}\mid\v{\ell})
=\int \int q(\vx)\,C_{\ell|X}(\vx, \vx')\,q(\vx')\,\ud \vx\,\ud \vx'
. \label{eq:var_inty}
\end{align}
Here we've used the following dense notation for the standard \gp expressions for the posterior mean $m$ and covariance $C$ respectively:
\begin{align}
m_{\ell|X}(\vx)& \deq k(\vx, X) k(X, X)^{-1} \v{\ell}\\
C_{\ell|X}(\vx, \vx')& \deq k(\vx, \vx') - k(\vx, X) k(X, X)^{-1} k(X, \vx').
\end{align}
If $q(\vx)= t(\vx)~\N{\vx}{\vmu}{\Lambda}$, $l_d=-\infty$ and $u_d=\infty$,  standard Gaussian identities \citep{BZMonteCarlo} can be used to give closed-form expressions for \eqref{eq:mean_inty} and \eqref{eq:var_inty}. Further, if $q(\vx)= t(\vx)~\N{\vx}{\vmu}{\Lambda}$ and $\Lambda$ is diagonal, \eqref{eq:mean_inty} and \eqref{eq:var_inty} are expressible using error functions for any arbitrary $l_d$ and $u_d$.
The use of an importance re-weighting trick (the use of $\tilde{q}(\vx) = \nicefrac{\tilde{q}(\vx)}{q(\vx)} q(\vx)$ for any $\tilde{q}(\vx)$) allows integrals with respect to arbitrary $q(\vx)$ to be approximated. 

The posterior variance of $\inty{\ell}$, \eqref{eq:var_inty}, provides a natural loss function for approximate inference: we follow \citet{osborne2012active} in Section \ref{sec:active} in minimising the expected variance to select observations. 

\section{Gaussian process modelling of likelihood functions}
Let's define $\ell\mx = \ell(x\mx)$ as the maximum observed likelihood, and likewise $\ell\mn = \ell(x\mn)$ the minimum. We aim to come up with a transform $f(\tl) = \ell$, $f\colon \reals \to \reals^+$ that is both monotonically increasing and has monotonically increasing derivative. 
Our transform $f$ must satisfy the constraints 
\begin{align}
 f(0) & = 0 \label{eq:f_constraint_1}\\
 f\biggl(\frac{\sqrt{\ell\mx}}{\gamma}\biggr) & = \ell\mx 
\label{eq:f_constraint_2}\\
 \beta\, f'(c\,\beta) & = a\,(\ell\mx - \ell\mn) \label{eq:f_constraint_3}\\
 \beta\, f'(d\,\beta) & = b\,(\ell\mx - \ell\mn). 
\label{eq:f_constraint_4}
\end{align}
Constraints \eqref{eq:f_constraint_1} and \eqref{eq:f_constraint_2} simply the set the scale of our latent transformed space; without loss of generality, we can set  
$\text{min}_{x}\,\tl(x)=0$. $\tl(x\mx)=\beta$ is left as a free parameter. 

We have $\sd\bigl(\tl(x)\bigr) \lesssim (\beta - 0)$: that is, the \sd of a \gp posterior at $x$ is typically less than the range of the observations on which it is conditioned. Hence, under our linearisation, the left-hand-side of \eqref{eq:f_constraint_3} is the approximate $\sd$ of $p\bigl(\ell(x)\bigl)$ for $\tl(x) = c \tl(x\mx)=c \beta$, some fraction of the maximum value for our set of observations. The right-hand-side expresses the constraint that this should be a pre-set fraction $a$ of the range of observed likelihoods. If we select $a\in \reals^+$ and $c\in \reals$ as small, this constraint is motivated by our desideratum that likelihoods have small mean also have small \sd. Similarly, if we select $b\in \reals^+$ and $d\in \reals$ as large, constraint \eqref{eq:f_constraint_4} is motivated by our desire that likelihoods with large mean have large \sd .


We choose
\begin{equation}
 f(\tl) = \gamma\,\tl^2.
\end{equation}
which trivially satisfies \eqref{eq:f_constraint_1}.


\begin{figure*}
 \psfragfig{figures/lik_v_tlik}
\psfragfig{figures/gps_lik_tlik}
\caption{}
\end{figure*}


\section{Quadrature propagation: Bayesian Quadrature for Gaussian Integrals}
\label{sec:active}

We now turn to the problem of selecting appropriate convolution 


\bibliography{bub}
\end{document}
